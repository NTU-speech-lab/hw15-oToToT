\documentclass[a4paper,11pt]{article}
\usepackage[left=1.2cm, right=1.5cm, top=1.6cm, bottom=1.3cm]{geometry}
\usepackage{xeCJK}
\usepackage{indentfirst}
\usepackage{tabularx}
\usepackage{float}
\usepackage{graphicx}
\usepackage{amsmath}
\usepackage{listings}
\usepackage{verbatim}
\usepackage{fancyhdr}
% \usepackage[compact]{titlesec}
\usepackage[usenames,dvipsnames]{xcolor}

\setCJKmainfont{NotoSansCJKtc-Thin}
\setmonofont{Consolas}

\definecolor{CodeGreen}{rgb}{0,0.6,0}
\definecolor{CodeGray}{rgb}{0.5,0.5,0.5}
\definecolor{CodeMauve}{rgb}{0.58,0,0.82}
\lstset{
    basicstyle = \ttfamily\footnotesize, 
    breakatwhitespace = false,
    breaklines = true,         
    commentstyle = \color{CodeGreen}\bfseries,
    extendedchars = false,
    keepspaces=true,
    keywordstyle=\color{blue}\bfseries, % keyword style
    language = C++,                     % the language of code
    otherkeywords={string},
    numbers=left,
    numbersep=5pt,
    numberstyle=\tiny\color{CodeGray},
    rulecolor=\color{black},
    showspaces=false,
    showstringspaces=false,
    showtabs=false,
    stepnumber=1,       
    stringstyle=\color{CodeMauve},        % string literal style
    tabsize=2,
}
% from https://blog.csdn.net/RobertChenGuangzhi/article/details/45126785

\title{Machine Learning 2020 - Homework 15 Report}
\author{學號:b08902100, 系級:資工一, 姓名:江昱勳}
\date{}

\begin{document}
\pagestyle{fancy}
\fancyhead[L]{Machine Learning 2020 - Homework 15}
\fancyhead[R]{Author: b08902100 江昱勳}

\maketitle

% \verbatiminput{HW2_S.txt}
% \lstinputlisting{HW2.cpp}

\begin{enumerate}
    \item Policy Gradient 方法
    \begin{enumerate}
        \item 請閱讀及跑過範例程式,並試著改進 reward 計算的方式。
        \item 請說明你如何改進 reward 的算法,而不同的算法又如何影響訓練結果?
        
        這裡我採用了discounted reward的方式改進reward的算法,並且每搜集64個episode的資料後,才去更新Actor,這裡discount的衰減率我設成0.9,最後total reward為131.4621654997108,final reward為0.5483643130777409。

        而原本沒有改進reward的方法total reward約為$-54.67351046366531$
    \end{enumerate}
    \item 試著修改與比較至少三項超參數(神經網路大小、一個 batch 中的回合數等),並說明你觀察到什麼。
    
    原始網路模型: 第一層線性層$8 \times 16$並使用$\tanh$作為活化函數、第二層線性層$16 \times 16$並使用$\tanh$作為活化函數,最後一層線性層$16 \times 4$並使用softmax作為活化函數,使其作為機率輸出。

    修改後的網路模型: 第一層線性層$8 \times 32$並使用$\tanh$作為活化函數、第二層線性層$32 \times 32$並使用$\tanh$作為活化函數,第四層線性層$32 \times 16$並使用relu作為活化函數,最後一層線性層$16 \times 4$並使用softmax作為活化函數,使其作為機率輸出。

    嘗試將網路從後發現model會變得更加的不穩定,從圖\ref{fig:pg_big}可以看到train了300個epoch後model還非常的浮動,到500 epoch後有稍微穩定一些,不過整體表現還是不好,推測是反而較不容易學到正確的知識,或許每次的episode要再加大一些。

    \begin{figure}[ht]
        \centering
        \includegraphics[width=.45\textwidth]{images/discount_big.png}
        \caption{Policy Gradient with larger model}
        \label{fig:pg_big}
    \end{figure}

    當初會採用0.9作為衰減率的原因是因為使用random agent玩了1000次後的平均episode長度為91.38,第一個epoch到最後一個epoch的差距為$0.9^{91.38}\approx 0.00006586892$,而每10\%左右的差距為$0.9^{9} \approx 0.387420489$,我自己估計這樣的結果是可以的。

    從圖\ref{fig:pg8}可以看到,實際上採用0.8作為衰減率訓練total reward約為-211左右,而final reward約為-100,可以發現結果也不盡理想,推測我們還是會希望採用更遠一點的操作,因為一次的失誤可能會需要很長一段時間的修正。

    \begin{figure}[ht]
        \centering
        \includegraphics[width=.45\textwidth]{images/rate8.png}
        \caption{Policy Gradient with discount factor 0.8}
        \label{fig:pg8}
    \end{figure}

    若訓練時每個epoch只採用8個episode時,模型在訓練時會變得較不穩定,可以從圖\ref{fig:ep8}看到,起伏非常的大,最後的結果也不好,total reward為-214.28588577033656,final reward為-100。並且從圖\ref{fig:res_ep8}也可以看到結果非常的差,直接翻車了。

    \begin{figure}[ht]
        \centering
        \includegraphics[width=.45\textwidth]{images/ep8.png}
        \caption{Policy Gradient with 8 episode each batch}
        \label{fig:ep8}
    \end{figure}

    \begin{figure}[ht]
        \centering
        \includegraphics[width=.45\textwidth]{images/result_ep8.png}
        \caption{Policy Gradient with 8 episode each batch}
        \label{fig:res_ep8}
    \end{figure}

    \item Actor-Critic 方法
    \begin{enumerate}
        \item 請同學們從 REINFORCE with baseline、Q Actor-Critic、A2C 等眾多方法中擇一實作。
        
        嘗試使用A2C的方式訓練模型遊玩此遊戲,每次抽取128次的episode拿來更新,訓練300 epoch,訓練的時候每次都同時用同一組episode的資料去更新actor跟critic,同時衰減率與前面相同我設成0.9,最後total reward為-165.2634328122486,final reward為-100。

        \begin{figure}[ht]
            \centering
            \includegraphics[width=.45\textwidth]{images/a2c.png}
            \caption{A2C}
            \label{fig:a2c}
        \end{figure}

        \item 請說明你的實做與前者(Policy Gradient)的差異。
        
        可以看到這次模型的結果不盡理想,推測是因為我同時訓練actor跟critic,在actor還沒學好的時候critic學到的內容自然也不盡正確,而policy gradient在這個遊戲上可能沒那麼困難,所以加入discount後就可以獲得不錯的結果。

        此外我也有嘗試先訓練actor 100epoch後再訓練critic 100epoch,再一同訓練actor + critic效果也不太好,甚至比原先效果更差,這部分我推測也有可能是因為在訓練critic的時候使用的資料主要仍是採用critic自己生成出來的部分,而不是來自環境,所以應該改善這部分。

        \begin{figure}[ht]
            \centering
            \includegraphics[width=.45\textwidth]{images/a2c2.png}
            \caption{Different training of A2C}
            \label{fig:a2c2}
        \end{figure}

    \end{enumerate}
    \item 具體比較(數據、作圖)以上幾種方法有何差異,也請說明其各自的優缺點為何。
    
    從圖\ref{fig:res_pg}可以看到,其實原先只採用policy gradient的方法在落地的時候就有不錯的表現了,不過其遊玩過程仍不太好;而改用0.9的衰減率後可以從圖\ref{fig:pg9}看到訓練過程大幅的提升,而結果也很不錯(圖\ref{fig:res_best})。其他方法我則幾乎沒有看到太多好的表現。最讓我訝異的部分是A2C的部分意外的差,完全找不出任何優點。
    
    \begin{figure}[ht]
        \centering
        \includegraphics[width=.45\textwidth]{images/pg.png}
        \caption{Policy Gradient}
        \label{fig:pg}
    \end{figure}

    \begin{figure}[ht]
        \centering
        \includegraphics[width=.45\textwidth]{images/result_pg.png}
        \caption{Policy Gradient}
        \label{fig:res_pg}
    \end{figure}

    \begin{figure}[ht]
        \centering
        \includegraphics[width=.45\textwidth]{images/best.png}
        \caption{Policy Gradient with discount factor 0.9}
        \label{fig:pg9}
    \end{figure}

    \begin{figure}[ht]
        \centering
        \includegraphics[width=.5\textwidth]{images/result_best.png}
        \caption{Policy Gradient with discount factor 0.9}
        \label{fig:res_best}
    \end{figure}

\end{enumerate}

\end{document}